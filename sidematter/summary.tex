%*******************************************************
% Summary
%*******************************************************
\pdfbookmark{Summary}{Summary}

\chapter*{Summary}
\label{chp:summary}

For organisms navigating a complex world in constant flux, visual motion is a fundamental cue. Any movement of a physical object results in a specific pattern of spatiotemporal correlation in reflected light.
%Correspondence between visual features at two points in space and time, for instance, commonly indicates movement from one location to the next.
By reliably extracting such regularities from optic signals impinging on appropriate sensory organs, an animal gains the ability to locate and identify objects, segregate foreground from background, determine depth, anticipate collisions, and estimate the motion of its own body in three-dimensional space. It is this multifaceted utility of motion cues that makes their representations particularly prevalent in a wide range of sensory apparatuses, ranging from the small, highly specialized nervous systems of insects up to the comparatively vast brains of primates. Wherever present, motion vision supports a multitude of intricate and critical behaviors, including locomotion, foraging, and mating.

The computation of motion is a non-trivial but well defined task which over many decades has generated considerable interest within the field of sensory neuroscience. As a consequence, direction-selectivity has become a canonical example of neural processing and offers a powerful test case for emerging tools of circuit analysis.

During my doctoral studies, I investigated motion vision in the model system \textit{Drosophila melanogaster}. Fruit flies are ubiquitous animals that exhibit a varied but conveniently stereotypical repertoire of behaviors. Among them is of course the ability to effectively move through their surroundings by walking or high-speed flight. \textit{Drosophila} utilize visual motion to stabilize and control these maneuvers. Through the computation of global optic flow fields, flies estimate current ego-motion and calibrate their motor instructions accordingly. An example for this type of motion-guided compensatory mechanism is the so-called optomotor response. When faced with global motion toward one side, flies tend to walk or fly in the same direction. This simple algorithm efficiently counteracts the perturbations that may result from, for instance, air turbulence. Early behavioral studies have yielded a compact algorithmic model of how direction-selectivity is achieved in the insect visual system. This detector is based on the cross-correlation of spatially separated, asymmetrically delayed luminance signals and closely recapitulates both insect behavior and response properties of globally motion-sensitive cells in the fly brain.

Due primarily to its history within the field of genetics, \textit{Drosophila} provides a rich set of genetic tools that allow activation, silencing, visualization, and functional imaging of targeted neuron types. In combination with classical behavioral and physiological techniques as well as high-throughput connectomics, this permits unprecedented access to the visual circuits computing motion.

I focused on two core research questions. First, novel techniques have put a cell-level map of the circuits implementing aforementioned detector model within reach. What are the first direction-selective stages within the fly visual system, what are their direct inputs, and what is the correspondence between neural elements and algorithm? Second, flies traverse cluttered and complicated visual surrounds. Realistic stimuli pose severe challenges. How then are the particular properties of natural scenes reflected in the algorithms and neural circuits that underlie motion detection in the fly visual system?

The findings of this work were published in four peer-reviewed articles that together form the cumulative thesis at hand.

In a first study, we identified a set of locally motion-sensitive cells in the \textit{Drosophila} optic lobe. Previous studies had shown that motion is processed twice in the fly visual system: once for positive contrast (ON) and once for negative contrast (OFF). Using calcium imaging from targeted neuron types, we demonstrated that four sub-types of T4 cells respond to localized ON motion in one of the four cardinal directions. Conversely, sub-types of T5 were sensitive to OFF motion in these particular directions. When we genetically silenced T4 or T5, motion responses to either ON or OFF stimuli were selectively abolished in downstream motion-selective cells of the lobula plate that receive input from T4 and T5. Finally, T4- and T5-silenced flies showed polarity-specific deficiencies in walking behavior when stimulated with competitive ON-OFF optomotor stimuli. This work further constrained the locus of motion computation and isolated the critical local direction-selective elements in the fly visual system.

Next, we investigated potential input elements to motion-sensitive T4 cells. Connectivity analysis based on electron microscope imaging of the medulla had suggested cell types Mi1 and Tm3 as likely inputs to the ON motion detector. Under this model, one neuron would represent the direct and the other the delayed arm of a cross-correlation detector as described above. By genetically silencing either Mi1 or Tm3, we were able to show that this circuit layout is incomplete. In line with model predictions, blocking Mi1 resulted in an ON-specific loss of motion sensitivity in cells downstream from T4 as well as a behavioral optomotor assay. However, when we silenced Tm3, only high-velocity stimuli were affected. At lower speeds, responses remained direction-selective. These findings strongly suggested that the underlying circuit scheme is more complex; in all likelihood, additional cell types are involved.

In a subsequent study, we characterized a comprehensive set of T5 input elements. Via calcium imaging, we established the spatial and temporal filter properties of cell types Tm1, Tm2, Tm4, and Tm9, which connectomic reconstructions had identified as the cells providing a vast majority of synapses to the OFF-selective motion detector. None of them were themselves selective for direction. This demonstrated conclusively that motion was computed within the dendrites of T5. Interestingly, temporal signatures were diverse and covered the full range from fast high-pass cells like Tm2 or Tm4 to pure low-pass characteristics in Tm9 via the intermediate Tm1. Such a filter bank is ideally suited to generating direction-selectivity. Additionally, we were able to show that silencing single cell types or pairs of cell types noticeably affected motion sensitivity as measured through electrophysiology or behavioral assays, ruling out the redundancy of any element.

Finally, having probed the neural underpinnings of \textit{Drosophila} ON and OFF motion vision, we related tuning properties of the pathways to their natural context. We evaluated the ability of flies to behaviorally match the velocity of rigidly translating natural images. While silencing T4 and T5 in conjunction abolished responses completely, individual blocks had no discernible effect on performance, indicating that both pathways were well adapted to the task of estimating scene velocity. Surprisingly, using electrophysiology and behavioral assays we could show that the ON pathway has starkly different tuning properties when compared to its OFF counterpart. The latter is tuned to much higher edge velocities. When we trained an \textit{in silico} model on the same estimation task, we recovered similar tuning asymmetries. From this, we concluded that neural circuitry in the \textit{Drosophila} visual system is precisely adapted to the statistics of natural scenes.