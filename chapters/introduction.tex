%************************************************
\chapter{Introduction}
\label{chp:introduction}
%************************************************

% Here, cite Dickinson (2014)

\section{Normative views of sensory neuroscience}

A core tenet of evolutionary theory is the idea that animals are in some sense adapted to their particular niche \citep{Darwin:1859aa}. This notion naturally extends to brains and specifically sensory systems. Ecological environments are not arbitrary; they exhibit statistical structure. The laws of physics, for instance, impose a level of regularity onto the set of possible sensory stimuli. Some such stimuli are more probable than others while certain configurations are outright incompatible with the rules that govern any given organism's surroundings.

It is a reasonable supposition that nervous systems strive to determine the veridical state of the world based on noisy sensory data, insofar as it supports a given behavioral program. \textit{A priori} assumptions about the way the world tends to be should then make this process more efficient and reliable. This approach to sensory systems enjoys a rich history, going back to Helmholtz who framed perception as the probabilistic problem of "unconscious inference" \citep{Helmholtz:1867aa}. It is closely connected to Bayesian theories of sensing but operates on much longer, evolutionary time scales \citep{Doya:2007aa}.

The evolutionary or teleological perspective is inherently normative. From a combination of purpose and environment, one can derive predictions about what properties the system ought to have if it is to fulfill its purpose effectively and efficiently. These predictions can then be tested experimentally. As such, it complements the purely descriptivist perspective on neural perceptual machinery which heavily relies on simplified, tractable inputs. A central theme of my dissertation is the question to what extent natural visual statistics are reflected in the properties of neural circuitry, using fly motion vision as a model system. The following section provides some relevant background.

\subsection{Statistics of natural visual scenes} Understanding the natural stimulus distribution is fundamental to normative approaches in sensory neuroscience. Given the topic at hand, this section focuses on natural visual stimuli. Whether captured by eye or camera, natural images exhibit clear regularities that set them apart from uniformly distributed noise. Realistic images thus represent only a small subset of all possible pixel configurations.

This is easily visualized by manipulating pictures in Fourier space. Unaltered images exhibit edges, gradients, homogeneous textures, and segregated objects. These typical features also extend to natural video sequences. Randomizing the phase structure of images scrambles their higher-order structure and yields phenomenologically atypical textures, even though the manipulation conserves second-order statistics as a consequence of the natural amplitude spectrum. Nonetheless, local patches may well resemble real image features. Finally, when one flattens the image's frequency spectrum, unambiguously artificial noise remains. Critically, at each stage we can clearly distinguish between images that fall within or outside the distribution of natural visual scenes.

% FIGURE!!!

Large corpora of calibrated natural scene images like the ones generated by \citet{vanHateren:1998jt} or \citet{Tkacik:2011aa} make it possible to characterize fundamental statistical properties. Below, I describe a subset of relevant features, ranging from first-order parameters to more complex traits.

\subsubsection{Luminance}

As a first step, one can measure the distribution of pixel luminance. After per-image normalization, linearly scaled luminance values in natural images are typically positively skewed \citep{Brady:2000aa,Laughlin:1981wn,Geisler:2008gu}. That is, pixels that are dark relative to average luminance numerically outweigh bright ones. When put on a logarithmic scale, this results in a symmetric distribution. The finding applies universally to image sets across a multitude of environments. Presumably, the skewness follows from basic physical principles.

\subsubsection{Contrast}
Another fundamental statistic is local contrast. Its quantification is less straightforward than in the case of luminance as it requires assumptions about the sensory system and the stimulus under investigation. Common definitions are the difference between feature and background luminance divided by the latter (so-called Weber contrast), the standard deviation of normalized luminance in a small image patch (so-called root-mean-squared contrast), or the difference of luminance extrema normalized by their sum (so-called Michelson contrast, often applied to wide-field stimuli).

Alternatively, in analogy to processing in retinal ganglion cells, contrast can be modeled as the response of divisively normalized center-surround receptive fields. Here, a similar asymmetry as for luminance emerges: At all spatial scales, natural images contain more dark (OFF) than light (ON) contrast \citep{Ratliff:2010kb,Cooper:2015in}. This is a direct consequence of the positively skewed luminance distribution. Luminance and contrast are not fully independent: their correlation is small but clearly negative \citep{Geisler:2008gu}.

\subsubsection{Spatial patterns}
Spatial structure represents one of the most informative aspects of any scene. A hallmark of natural images is the shape of their Fourier amplitude spectra \citep{Geisler:2008gu}. The average contribution of component frequencies falls with frequency and is well modeled by the function $1/f^n$ with $n\approx1.0$ \citep{Ruderman:1994ty,Field:1987ua,Dyakova:2015dy}. As a consequence, natural images are approximately scale-invariant; zooming in or out does not substantially affect the shape of the Fourier spectrum. Interestingly, the pink noise model also reproduces spatial properties of local patches in naturalistic video sequences \citep{Dong:1995aa} even though the complex statistics of animal movement make the task of gathering ecologically relevant stimuli difficult.

As discussed above, pink Gaussian (or $1/f$) noise serves as a reasonable local approximation to realistic images but falls short in several ways. One of them is a linearly symmetric luminance distribution which does not reproduce natural skew \citep{Geisler:2008gu}. Moreover, it fails to model the heavy-tailed response properties of arbitrary receptive fields scanning natural scenes \citep{Field:1987ua}. Clearly, the Fourier spectrum does not provide an exhaustive description of real-world spatial features. Natural images show pronounced co-linearity, parallel contours, sharp transitions, and many other forms of spatial regularity that go beyond simplistic second-order features.

\subsubsection{Other features}
In addition to the most salient subset of features outlined above, researchers have mapped the natural statistics of many additional image properties. They include depth \citep{Huang:2000aa}, color \citep{Ruderman:1998aa}, and optic flow \citep{Roth:2005aa,Roth:2007bg}, but are often limited in throughput by the available measurement devices and sensors \citep{Geisler:2008gu}. Reliable estimation of natural statistics in high-dimensional spaces requires sufficient volumes of data. For this reason, static but large image databases have been an important foundation for research in the space \citep{vanHateren:1998jt}.

\subsubsection{Visual ecology}
To relate any visual system to naturally occurring stimuli, a reasonable approximation of the neuro-ecologically relevant image distribution is required. Available natural scene libraries are biased toward human visual surrounds when it comes to choice of environment, perspective, focal length, resolution, and other parameters \citep{Tkacik:2011aa}. Many of these qualities are at odds with the experience of a typical fruit fly. For instance, the \textit{Drosophila} eye processes scenes at the comparatively low resolution of approximately $25 \times 25$ facets or "pixels", covering almost $180\degree$ of azimuth at a separation of $\approx5\degree$ \citep{Borst:2009gv}. The spatial acuity of the fly eye is thus orders of magnitude below that of its human counterpart. Moreover, not much is known about the ground-truth visual statistics of the environment in which \textit{Drosophila} initially evolved \citep{Dickinson:2014aa}.

%Not a problem because scale invariance.
Spatial differences between typical inputs, however, are attenuated by the aforementioned invariance of realistic image amplitude spectra. Generally, drosophilids are extremely widespread and resilient organisms. Their brains possess comparatively few neurons, resulting in limited degrees of freedom. As a consequence, visual adaptation is unlikely to be deeply environment-specific \citep{Dickinson:2014aa}. Finally, many of the lower-order statistics discussed above appear to be due to fundamental properties of the world and therefore generalize to visual environments across the board \citep{Geisler:2008gu,Simoncelli:2001dn}. \citet{vanHateren:1997vg}, for instance, gathered natural visual time series data simply by walking through forests while recording the output of a simple LED attached at eye level.

\subsection{Information theory}

\subsection{Efficient coding}
Sensory systems evolve under a multitude of constraints \citep{Sterling:2015aa}. One of them is the need to detect stimuli that are survival-critical. Another is the need to perform this task using a minimum of metabolic energy. A third comes from the natural distribution of environmental features. The \textit{efficient coding hypothesis}, going back to \citet{Attneave:1954aa}, \citet{Barlow:1961aa}, and others, formalizes this approach to understanding biological sensors. Their theory was heavily influenced by the development of information theory which made a rigorous quantification of notions like channel capacity or redundancy feasible \citep{Cover:2006aa,Borst:1999hw}.

Efficient coding assumes that the goal of a sensory system is to represent as much relevant information as possible using the smallest feasible amount of resources. Selecting appropriate objective functions is, of course, fraught with difficulty. Ground truth constraints are generally not available and sensory organs often support a broad range of behavioral functions, each of which may necessitate a different definition of relevance. Nonetheless, the theory has successfully predicted features of real systems by assuming basic, tractable goals. In the sensory periphery, this usually takes the form of general preservation of information, thus maximizing the number of possible downstream use cases.

The main target of early vision then becomes reduction of redundancy. Ideally, signals carried by peripheral sensory neurons should be statistically independent in order to minimize wasteful duplication of information. Natural images exhibit regular statistics such as characteristically shaped power spectra that give rise to specific correlation structures. By removing such correlations and emphasizing deviations from expected natural statistics, an operation commonly termed whitening, early vision minimizes energy expenditure while conserving features of the stimulus that are presumed to be behaviorally relevant.

The fly retina provides classic demonstrations of this principle at work. \citet{Laughlin:1981wn} measured naturally occurring luminance distributions and compared the resulting histograms to corresponding response functions of lamina bipolar cells which form the first processing stage after the light-sensitive photoreceptor. In line with predictions from efficient coding, these response functions effectively equalized the histograms, making all outputs equally likely under the assumption of a natural stimulus distribution. A related study \citep{Srinivasan:1982uq} could show that lateral inhibition in the fly retina reliably removes the long-range correlations typical for natural images, effectively suppressing background, retaining sensitivity to small fluctuations, and implementing a type of predictive coding.

Constraint triples of this type---minimization of resources, maximization of transmitted information, assuming some naturalistic stimulus distribution---have been equally fruitfully applied to early processing in retina and visual cortex of mammals. Center-surround receptive fields in both retina and lateral geniculate nucleus (LGN) of the cat have been suggested to implement spatial filters that are well suited to whitening the typical power spectra of natural images \citep{vanHateren:1992aa,vanHateren:1993aa,Atick:1992aa}. \citet{Dan:1996aa} confirmed this prediction experimentally by recording LGN responses to natural movies, finding them to be largely statistically independent. \citet{Ratliff:2010kb} argue that the asymmetry in ON and OFF contrast prevalence mentioned above explain the difference in numbers between ON and OFF retinal ganglion cells in the vertebrate retina. Interestingly, evidence from primary sensory neurons in V1 of awake mice indicates that adaptation to natural scene statistics partially depends on experience; if raised in stimulus-deprived environments, predictive coding of specifically real images is abolished \citep{Pecka:2014aa}.

A closely related normative doctrine is that of sparse and distributed coding: the idea that sensory systems like visual cortex aim to represent natural stimuli using a minimum of active neurons \citep{Simoncelli:2001dn}. \citet{Ohlshausen:1996aa}, for instance, optimized a linear generative model to reconstruct natural images under the constraint of activation sparseness. The resulting filters bear striking resemblance to spatial receptive fields in area V1, indicating that early visual cortex is adapted to the task of efficiently representing real-world stimulus distributions \citep[for a related method, based on independent component analysis, see][]{vanHateren:1998jt,Bell:1997ve}.

\subsection{Task-driven optimization}
Efficient coding theory sidesteps the question of task relevance and presupposes that peripheral sensory systems perform lossless compression while maximizing efficiency. This has been a frequent source of criticism \citep{Simoncelli:2003aa}. After all, brains solve particular problems, so not all information is equal. Relevance may well depend on the particular nature of downstream processing or even behavioral state. For this reason alone, efficient coding is unlikely to scale to higher-level computation.

Instead of choosing a generic normative aim like information preservation, one may be able to do better by picking a specific, task-bound objective function. Encouraging examples come from recent advances in artificial pattern recognition \citep{Bishop:2006aa} and specifically the hierarchical models prevalent in so-called deep learning \citep{Goodfellow:2016aa}. Response properties along the visual cortical pathway go from simple local receptive fields in V1 to object-specific and invariant representations in higher areas \citep{Felleman:1991aa,Yamins:2016hg}. Such neurons are sensitive to, for instance, cars regardless of perspective. Hierarchical neural networks that mimic aspects of this organization have reached human-like performance on large object recognition data sets, made possible by advances in optimization techniques and raw processing power \citep{LeCun:2015dt,Fukushima:1980ve}. \citet{Yamins:2014gi} modeled an artificial deep network after the primate object recognition cascade consisting of areas V1, V2, V4, and IT. After training this system to recognize classes of objects in natural images, they compared learned weights with representations in the biological system and found striking similarities, at least in higher layers \citep{Cadieu:2014in}. More generally, early stages of visually trained deep networks often exhibit receptive fields that resemble those found in the vertebrate retina or V1 \citep{Yamins:2016hg}.

In psychophysics, several studies have successfully predicted texture salience from statistical properties of natural images \citep{Tkacik:2010aa,Hermundstad:2014aa}. By determining maximally informative features in real stimuli, it was possible to predict behavioral performance for a given synthetic texture.

Task-driven approaches have also been applied to the visual system of the fly. \citet{Clark:2014aa} and \citet{Fitzgerald:2015aa} optimized a motion detector to maximize the linear correlation between time-averaged model output and true velocity of rigidly translating natural images \citep[see also][which is part of this thesis]{Leonhardt:2016ex}. While functionally plausible, this makes strong assumptions about the true goal of motion-sensing elements. Other studies put fly motion detectors in the functional context of closed-loop course stabilization but did not take natural stimulus distribution into account \citep{Warzecha:1996bm,Warzecha:1998tn}.

Combining behaviorally relevant targets with biologically plausible models appears to be a promising tool for understanding neural processing beyond the sensory periphery. Of course, for complex and highly multiplexed information processing systems like the brain, ascribing goals remains challenging: one may well be wrong about what any given circuit is in fact trying to achieve. Additionally, the approach critically depends on the choice of model---say, a feedforward network for cortical processing as opposed to a more plausible recurrent system---and the techniques used for post-hoc comparisons between model and neural circuit \citep{Yamins:2016hg}.

% Bialek H1 thing-y
% van Hateren work

% Dependency on ecology (van Hateren work, Dickinson ecology thing)
% Figure: three images (Gaussian noise, 1/f noise, natural image)

% \section{ON and OFF pathways in sensory processing}

% \section{Motion vision in the fly}

% \subsection{Motion-guided behaviors}
% \subsection{Visual stimuli}
% \subsection{Algorithmic models}

% \section{Structure and function of the fly visual system}
% \subsection{Retina}
% \subsection{Lamina}
% \subsection{Medulla and lobula}
% \subsection{Lobula plate}
% \subsection{Downstream circuits}

% \section{Experimental tools in circuit neuroscience}
% \subsection{Neurogenetics}
% \subsection{Behavioral assays}
% \subsection{Electrophysiology}
% \subsection{Imaging techniques}
% \subsection{Connectivity analysis}

% \section{Concluding remarks}