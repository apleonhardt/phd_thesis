%************************************************
\section{Functional Specialization of Neural Input Elements to the Drosophila ON Motion Detector}
% \section{Ammer et al., 2015}
\label{sct:manuscript_ammer}
%************************************************

In this study, we investigated the functional significance of two major input elements to T4 cells, Mi1 and Tm3. The paper was published in \textit{Current Biology} in July 2015.

\paragraph{Summary}
An electron microscope-based analysis of presynaptic T4 connectivity had previously revealed two numerically dominant inputs: columnar medulla cells Mi1 and Tm3. A marginal spatial offset between the two projection fields supported a model in which Mi1 and Tm3 implement the two arms of a Reichardt-type motion detector, one transmitting fast visual signals from one retinal location and the other relaying delayed input from a slightly offset position. We tested this hypothesis by silencing either Mi1 or Tm3 and assaying motion sensitivity in either lobula plate tangential cells or flies walking on a treadmill. Interestingly, while the loss of Mi1 activity selectively abolished responses to ON motion, as had been predicted from aforementioned model, the phenotype of Tm3 silencing was limited to ON stimuli traveling at high velocities. These findings were in disagreement with the suggested model and strongly hinted at further complexity and functional specialization in the T4 input structure.

\paragraph{Authors} Georg Ammer, \textbf{Aljoscha Leonhardt}, Armin Bahl, Barry J. Dickson, and Alexander Borst.

\paragraph{Contributions}
G.A.\ and A.\ Borst designed the study. G.A.\ performed electrophysiological experiments and anatomical characterization of expression patterns, analyzed the data, and wrote the manuscript with the help of A.\ Borst, \textbf{A.L.}, and A.\ Bahl. \textbf{A.L.}\ and A.\ Bahl performed behavioral experiments and analyzed data. B.J.D.\ generated SplitGal4 fly lines and hosted G.A.\ for characterization of Gal4 lines. A.\ Borst performed computational modeling.

\cleardoublepage

\includepdf[pages=2-last,scale=0.9,offset= 0 40,pagecommand={\thispagestyle{plain}}]{papers/ammer2015_plus_supplement}