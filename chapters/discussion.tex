%************************************************
\chapter{Discussion}
\label{chp:discussion}
%************************************************

% Section on DEVELOPING TECHNIQUES: voltage imaging, optogenetics
% Section on WHERE DOES POLARITY SELECTIVITY ARISE
% Section on DOWNSTREAM PROCESSING
% Maybe bring in MM drawings?
% Bring about depth bullshit
% Write about how the RD may be good ENOUGH
% Write about higher-order stuff

Fly motion vision provides a compelling example for the methods and goals of systems neuroscience. To extract optic flow from reflected light, the nervous system needs to perform non-trivial but clearly circumscribed computations. Flies accomplish the task using a small number of neurons and within few synapses, suggesting that the project of delivering a circuit-level description of direction-selectivity is indeed tractable. Algorithmic models give computational context in which we can embed and to which we can compare circuit schemes derived from experimental work. Additionally, motion is a critically relevant stimulus for animals in virtually all ecological niches. Optic flow provides information about our own movement, depth in a visual scene, as well as the movement of conspecifics, prey, and predators. The fact that motion represents such a fundamental cue allows us to put our models of the circuit in the functional context of defined goals.

In the course of this cumulative thesis, my collaborators and I have made substantial progress toward neural models of motion detection in the fruit fly \textit{D. melanogaster}. First, we were able to identify cell groups T4 and T5 as the direction-selective output elements of the ON and OFF motion pathways in the fly optic lobe \citep{Maisak:2013kk}. They form a retinotopic map that delivers locally motion-sensitive signals to the wide-field tangential cells of the lobula plate. Two major functional divisions emerged, one separating contrast polarities and the other concerning directions in visual space. T4 responds only to ON motion defined by brightness increases and T5 only to corresponding OFF motion defined by brightness decreases. Four sub-types of each are selective for only one of the four cardinal directions. When T4 or T5 were silenced, downstream responses both in tangential cells and walking flies were affected in a polarity-specific fashion. In conjunction with the finding that combined silencing of T4 and T5 abolishes all wide-field motion responses, this indicated that the two cell arrays are the dominant source of motion information in the fly brain.

Second, we investigated medulla elements feeding into the T4 pathway \citep{Ammer:2015jo}. Dense reconstruction had suggested a circuit layout where Mi1 and Tm3 represent the two arms of a Reichardt-type motion detector. Contrary to predictions from this model, only silencing of Mi1 abolished motion responses in tangential cells. Inactivation of Tm3 only had an effect in the specific stimulus regime of fast velocities. Behavioral work confirmed the findings, which ruled out the Mi1-Tm3 model and indicated increased neural complexity. Third, we explored this type of architectural complexity in the context of the T5 pathway where we studied the response properties and functional roles of input elements Tm1, Tm2, Tm4, and Tm9 using imaging and genetic silencing \citep{Serbe:2016ew}. They provide a broad spectrum of temporal and spatial filters to T5 which are well suited to the computation of motion under a Reichardt-type model. Critically, none of them are themselves direction-selective. When inactivated, physiological and behavioral phenotypes showed that all play a role in OFF motion detection. 

Fourth and finally, having established some of the neural basis of motion detection, we related the emerging two-pathway architecture to its functional context \citep{Leonhardt:2016ex}. In both behavior and physiology, we discovered substantial asymmetries in temporal tuning between the ON and the OFF channel. Simulation work suggested that these asymmetries constitute an adaptation to the particular demands of natural visual statistics.

\section{Neural models for motion detection}
Key impetus for the projects I pursued during my doctoral studies was mapping algorithmic elements onto concrete implementation in the form of neural elements. The Reichardt detector and its elaborations have been exceedingly successful at accounting for input-output relationships. Even detailed aspects of optomotor response and neural properties of tangential cells are well predicted by a simple combination of linear filtering and elementary mathematical operations \citep{Borst:1989vp,Borst:2002iw}. It was an open question whether this simplicity would be reflected by neural circuitry. In this section I discuss the mapping particularly in light of more recent developments.

\subsection{Input lines}
Standard models of local direction-selective units are based on two spatially separated inputs that filter visual signals asymmetrically. Work on T4 and T5 inputs, including ours, has hinted at surprising complexity in the presynaptic structure of fly elementary motion detectors. No obvious one-to-one correspondence between algorithm and circuit. So how should we map neural elements in the medulla onto these algorithmic inputs?

\subsubsection{ON pathway}
Based on connectivity and a limited spatial offset between projection fields that correlated with the preferred direction of the targeted cell, \citet{Takemura:2013ea} had proposed a two-arm model in which Mi1 and Tm3 relay visual input to the dendrites of T4 where motion is then computed through a correlation-type mechanism. Electrophysiological recordings from cell bodies of these two cells constrained the model further as the estimated time constant of a filter fit to Mi1 responses was somewhat larger than that of Tm3 \citep{Behnia:2014jh}. Neither Mi1 nor Tm3 were already selective for direction. Together, these findings predicted that input signals are combined on T4 dendrites in a non-linearly opponent fashion as in the model proposed by \citet{Barlow:1965aa}. The correspondence between circuit and a subunit of the Reichardt model would then have been almost one-to-one: starting from photoreceptors, the L1-Mi1 and L1-Tm3 pathways carry a slow and a fast signal, respectively, to the non-linearity implemented by T4.

Several factors detracted from the plausibility of this model. First, the reported difference between filter peaks was approximately \SI{18}{\milli\second} and thus exceedingly small compared to standard values used for modelling of tangential cell responses \citep{Behnia:2014jh}. The steady-state frequency optimum of a simple low-pass Reichardt detector is given by $(2 \pi \tau)^{-1}$ which would require $\tau \approx \SI{150}{\milli\second}$ for the typical peak at \SI{1}{\hertz} \citep{Maisak:2013kk}. While a direct transfer between time constant and delay difference is not trivial, particularly when the measured filter function is of higher order, the gap is still large. A quantitative model proposed by \citet{Behnia:2014jh} was able to replicate a well-defined optimum at \SI{1}{\hertz} due to the high-pass characteristics of the measured filters and, importantly, due to the subtraction of oppositely tuned units. Any circuit based on small differences between large signals, however, suffers from a lack of noise robustness. Moreover, the dendrites of T4 already appear to be highly direction-selective \citep{Maisak:2013kk} but there is now substantial evidence that the subtraction stage is implemented downstream of T4 \citep[see][and the sections below]{Mauss:2015kj}. Second, the reported separation of the centers of mass of Mi1 and Tm3 projection fields was on the order of \SI{1}{\degree} in visual space, corresponding to only \SI{20}{\percent} of inter-ommatidial distance. While this separation is sufficient to generate direction-selectivity, it again negatively impacts the signal-to-noise characteristics of the resulting circuit. Third, the circuit model clearly predicts that silencing of one input line should abolish direction-selectivity fully. This was not borne out by our findings \citep{Ammer:2015jo}; only inactivation of Mi1 affected downstream ON motion responses across the full range of tested stimuli. Note, however, that \citet{Strother:2017aa} found a more completely abolished grating response when imaging T4 in Tm3-silenced flies. Nonetheless, the available evidence pointed toward a more complex circuit layout.

Further studies have recently filled in some of the gaps in our understand of medulla circuitry feeding into T4. The reconstruction effort that had suggested the two-arm Mi1-Tm3 model was subject to methodological constraints that led to an incomplete connectivity matrix. In particular, not all processes impinging on T4 dendrites were followed to their originating columns.

Subsequent work used focused ion beam scanning electron microscopy (FIB-SEM) to image and reconstruct a full cartridge along with its six adjacent columns at a superior voxel size of approximately \SI{10}{\nano\meter} \citep{Takemura:2017aa}. In the resulting circuit diagram, Mi1 and Tm3 were confirmed as major inputs to T4 that jointly represent $\approx\SI{50}{\percent}$ of synapses. The spatial shift between projection fields could not be replicated. Several additional numerically relevant inputs were identified, chief among them Mi4 and Mi9 (complemented by C3, CT1, TmY15, as well as other T4 cells). Dendritic trees of T4 have an elongated structure that covers multiple columns of the medulla and whose orientation correlates with the lobula plate layer to which the sub-type projects. Intriguingly, while Mi1 as well as Tm3 projections tend to target the central area of the dendrite, both Mi4 and Mi9 form synapses in a spatial pattern that depends on the preferred direction of the T4 cell. For upward-sensitive T4c cells, for instance, Mi4 connects primarily on the dorsal end while Mi9 does so ventrally. This layout and in particular the separation of projection fields between Mi1/Tm3 and Mi4 or Mi9 lend themselves well to Reichardt-type motion computations.

Calcium imaging from these additional medulla cells has critically added to the purely structural view of the T4 circuit \citep{Strother:2014aa,Arenz:2017aa,Strother:2017aa}. As in the OFF pathway, neither of the four inputs is direction-selective by itself which confirms that T4 dendrites are the locus where motion is first extracted. \citet{Arenz:2017aa} used white-noise stimuli to map spatiotemporal receptive fields and found two transient units which were well-approximated by high-pass filters (Mi1 and Tm3) as well as two tonic units resembling low-pass filters (Mi4 and Mi9). Measurement of step responses yielded comparable results \citep{Strother:2017aa}. An interesting complication arises from the response sign of Mi9. While all other cells increase their calcium levels in response to ON stimulation, Mi9 is activated by OFF stimulation instead. In terms of connectivity, this finds a convenient explanation in the fact that Mi9 lies downstream of OFF-implicated lamina monopolar cell L3. It is feasible that the synapse connecting Mi9 and T4 effectively reverses the response sign, thereby providing an ON-like signal to T4. Taken together, the filter bank offers a much larger range of temporal properties than what \citet{Behnia:2014jh} had put forward, with time constant differences reaching hundreds of milliseconds. A broad spectrum of course then greatly simplifies the construction of highly direction-selective units.

Given that signals from Mi1 and Tm3 target overlapping parts of the central T4 dendrite and largely come from the same central cartridge, it is a distinct possibility that they interact to form a single functional input arm. There are at least three lines of evidence additionally supporting this notion. First, the FIB-SEM connectome indicates that Mi1 is itself pre-synaptic to Tm3. In fact, Mi1 is the numerically strongest Tm3 input behind just L1. Second, \citet{Strother:2017aa} performed optogenetic activation experiments using Chrimson to test functional connectivity between candidate medulla cells and T4 \citep{Klapoetke:2014aa}. Intriguingly, while the isolated activation of Mi1 or Tm3 only had negligible effects on calcium activity of T4 cells, joint excitation of the cell pair resulted in significant signals that were non-linearly amplified over the simple sum of individual responses. Third, our blocking experiments could show that Tm3 plays a critical role in ON motion detection when edge velocities were at the higher end of tested velocities. It is conceivable that Tm3 serves to shape and possibly sharpen signals emanating from the central portion of the visual field in concert with projections via Mi1. This may then only generates clear phenotypes when input dynamics are fast. Overall, the observed complexity highlights that the mapping from circuit to algorithm does not have to be one from neurons to filters and input lines. Individual algorithmic components could well be implemented by a group of neurally segregated units.

\subsubsection{OFF pathway}
Our work on OFF pathway elements paints a similar picture as the one that has now emerged for the ON counterpart. Tm1, Tm2, Tm4, and Tm9 jointly account for a vast majority of the input synapses onto T5 \citep{Shinomiya:2014dx}. None of them are direction-selective, which confines the critical computation to the dendrites of T5 \citep{Maisak:2013kk,Serbe:2016ew}. As with Mi1, Tm3, Mi4, and Mi9, they have varying filter properties ranging from the slow and tonic (Tm9) to the fast and phasic (Tm2 and Tm4) with Tm1 as an intermediary. A Reichardt detector using, for instance, Tm9 and Tm2 as delayed and direct line, respectively, exhibits high direction-selectivity and a frequency optimum in the physiologically plausible range. Conversely, some combinations like Tm2 and Tm4 provided little directional signal, making them unlikely candidates for inputs to the motion detector.

Interestingly, while agreeing on delay direction, we observed a much larger difference between the temporal response dynamics of Tm1 and Tm2 than what \citet{Behnia:2014jh} had reported previously. Possible reasons include calcium kinetics that exaggerate voltage timing differences or asymmetries between measurements in cell bodies and terminals, but note that the Tm2 step calcium responses we found are indeed slightly faster than the equivalent voltage deflection.

In our measurements, spatial receptive fields of all T5 inputs were isotropic, retinotopic, and small, with separation and half-width approximately corresponding to what was expected from facet layout. Additionally, all exhibited lateral inhibition; responses to large stimuli were suppressed. This was later corroborated by filter estimates derived from white-noise responses \citep{Arenz:2017aa}. In contrast, \citet{Fisher:2015aa} employed reverse-correlation and determined receptive fields for Tm9 whose extent was in excess of \SI{60}{\degree} in both elevation and azimuth. The reason for this drastic discrepancy remains unclear. Studies from our laboratory used a different GAL4 line to target Tm9. However, temporal tuning measurements as well as phenotypes in Tm9-silenced flies were in agreement across studies, casting doubt on this explanation. An interesting feature we found when establishing the size tuning of Tm9 using flickering bars of various sizes was an increase in response strength when the bars became large enough to resemble full-field flicker. Through some global pooling mechanism Tm9 cells appear to have access to information from a significant portion of visual space. This observation may be a first step toward reconciling the measurements if we assume that the measurements by \citeauthor{Fisher:2015aa} were performed in a way that would affect the global properties of the stimulus. For instance, if the recorded terminals have receptive fields close to the borders of the retinotopic map, asymmetric lateral signals may lead to a broadening of the input field of Tm9. From an algorithmic point of view, however, it remains unclear how true wide-field input would critically contribute to direction-selectivity in T5.

The strength of the behavioral phenotypes we found using physiological and behavioral measurements correlated distinctly with the number of synaptic contacts between the respective cell and T5 dendrites. Critically, all four blocks had an impact on downstream motion responses. Such a lack of redundancy does not indicate a simple division of labor between the potential input arms of the OFF motion detector. In contrast to our work on the ON pathway, velocity tuning curves did not reveal velocity-dependent functional specialization; the reduction in OFF response strength was generally conserved across stimulus frequencies. The strongest effects resulted from blocking Tm2 and Tm9 either individually or in combination. A simple conclusion from this would be to propose Tm2 as the fast and Tm9 as the slow arm of a elementary motion detector. However, this does little to explain the contribution of Tm1 or Tm4 which, particular in combination, also produced substantial phenotypes.

\subsubsection{Biophysical origin of delays}
So far, I have tacitly assumed that the temporal filtering of signals reaching T4 is purely intrinsic to the input cells. Given the substantial variety observed at the level of medulla output lines, this is a reasonable assumption.

It is currently not fully understood how medulla cells generate and tune their particular filtering properties. First, it is possible that filter properties are simply inherited from upstream lamina cells. This accounts for a significant fraction of the observed variability. In the OFF pathway, high-pass units Tm1, Tm2, and Tm4 all receive input from the transiently responding L2 \citep{Fischbach:1989uw,RiveraAlba:2011dd,Takemura:2017aa}, with L4 additionally connecting to Tm2. Tonic Tm9 cells, on the other hand, are primarily post-synaptic to L3 for which slow kinetics have been demonstrated \citep{Silies:2013jp}. Within the ON pathway, transient L1 projects to band-pass cells Mi1 and Tm3 while tonic Mi9 lies downstream of L3. Mi4 is targeted by L5 for which photoreceptor input originates from reciprocal connections with L1 (but note that little is known about the intrinsic tuning of L5). Under this scheme, the medulla filter bank is generated by summing lamina output kinetics in various configurations which provides numerous degrees of freedom. Lamina cells then act as building blocks from which more varied filters can be derived, which attributes interesting functional significance to the layered structure of the fly optic lobe. However, while basic characteristics appear to be derived from upstream processing, further multiplexing occurs. Tm1 and Tm4, for instance, exhibit differing kinetics despite their shared main input L2.

Second, cell-intrinsic mechanisms in medulla pathways could further refine temporal properties. Passive, purely electrotonic properties of the membrane in neural "cables" produce effects like signal attenuation along appropriately constructed neurites. This results in delays and low-pass filtering of voltage signals that depend on the geometry of processes \citep{Koch:2004aa}. Active conductances along the path may additionally shape signals through, say, non-linear amplification or activation cascades that introduce temporal offsets. Moreover, synapses represent junction points at which elaborate signal modifications are implemented through transmission machinery. High-pass filtering, for instance, resembles adaptation. If a synapse removes the long-term mean of the signal by rapid habituation, only sensitivity to fast changes remains. By modulating the kinetics of this adaptation process, different high- or band-pass characteristics are achieved. Adaptive mechanisms have previously been employed to explain phasic output in lamina monopolar cells \citep{Laughlin:1978aa}. Alternatively, high-pass characteristics also emerge when subtracting differentially low-pass filtered signals. This offers yet another biophysically simple mechanism for rendering output transiently sensitive.

It is crucial to note that not all temporal filtering has to be present in the output of medulla inputs to the dendrites of motion-sensitive cells. Indeed, the synaptic apparatus connecting pre-synaptic cells and T4 or T5 may plausibly contribute to the required asymmetric filtering. For instance, there is now evidence from RNA profiling of isolated T4 and T5 cells that these cells express both ionotropic and metabotropic variants of acetylcholine receptors \citep{Shinomiya:2014dx,Pankova:2016aa}. If cholinergic input from one spatial location triggers a slow, muscarinic version and the other a fast, nicotonic version, resulting timing differences may be sufficient to permit the disambiguation of motion direction. More complete neurotransmitter profiles of medulla cells are now available: \citet{Shinomiya:2014dx} propose that all four T5 inputs are cholinergic while findings by \citet{Takemura:2017aa} suggest that Mi1 and Tm3 are cholinergic, Mi4 GABAergic, and Mi9 glutamatergic. This variety offers substantial leeway for synaptic implementations of temporal filtering. Additionally, it is entirely possible that a combination of cell-intrinsic and synaptic mechanisms gives rise to the temporal input profile. Note, however, that the measured cell-intrinsic delays of certain medulla cell combinations appear sufficient to generate strong direction-selectivity \citep{Arenz:2017aa}.

\subsection{Nature of the non-linearity}
The comprehensive mapping of medulla input cells along with the finding that T cell neurites targeting the lobula plate are already selective for direction had revealed T4 and T5 as the locus of the non-linear interaction underlying motion detection. However, the details of this computation remained elusive.

Detailing the neural implementation of a multiplication-like operation has the potential to clarify functional segregation of pre-synaptic elements. After all, input cells clearly outnumber input lines of the Reichardt detector in both pathways. With the exception of velocity-dependence for Tm3, none of the silencing experiments had pinpointed particular functional division among the multiplexed input structure of T4 and T5. The neural non-linearity then further constrains required pre-synaptic layout.

\subsubsection{Dual mechanisms}
Subunits of the Reichardt detector and Barlow-Levick schemes share the basic principle of differentially delaying spatially adjacent signals and then comparing them to establish temporal order. Where they differ is the choice of non-linearity that implements comparison. For the Reichardt detector, this is correlation as realized by mathematical multiplication. Large output signals occur when the two inputs are in phase after one of them is delayed. For Barlow-Levick detectors, the essential operation is an AND-NOT gate as realized by, for instance, divisive inhibition. Only inputs with appropriate phase relationships pass; signals resulting from null motion are vetoed. The two layouts make clear predictions about the placement of the delay as well as the input-output signature of the operation that combines inputs. One of them is fundamentally facilitating; the other one is fundamentally suppressive.

Several studies have now made progress toward a neural understanding of the operations implemented by fly motion-sensitive neurons. First, \citet{Fisher:2015jo} used apparent motion to discriminate excitatory and inhibitory interactions on dendrites of genetically singled-out T4 and T5 cells. When stimulated with spatially neighboring, temporally separated flashes along the cells' preferred direction, T4 dendrites showed direction-selective calcium increases that exceeded the linear prediction calculated as the sum of responses to isolated flashes. The same facilitating response was observed for T5\footnote{Interestingly, \citet{Fisher:2015jo} also reported a non-linear suppression of responses to apparent motion along the null direction but discarded this as motion-unselective adaptation.}. This indicated that the non-linear interaction on both T4 and T5 dendrites is based on the amplification of coincident signals, in line with what the Reichardt model had predicted.

Second, \citet{Leong:2016hu} used systems identification to fit T5 white-noise responses. Given the inherent non-linearity of direction-selective responses, they built a cascade model that consisted of chained linear (L) receptive fields and point non-linearities (N), forming a LNLN feed-forward model that also modeled calcium binding dynamics. This phenomenological model contrasts with mechanistic accounts like the Reichardt detector. As expected and in line with the internal structure of motion energy models, response variance was best explained by a spatiotemporally slanted receptive field combined with polynomial non-linearities of orders 3 and above. Within the linear spatiotemporal filter, two spatially separated sub-fields could be identified---one with positive sign, one with negative sign. The authors interpreted this as a dual mechanism for generating OFF-specific direction-selectivity that recruits both facilitation and suppression, but note that their method does not allow for conclusive disambiguation of, say, ON facilitation and OFF suppression.

Third and finally, \citet{Haag:2016cq} probed the dendritic non-linearity through a combination of precisely targeted apparent motion and layer-specific read-out of T4 activity. The former was accomplished by telescopic stimulation of precisely isolated neuro-cartridges; the second by expressing the calcium indicator under the control of a T4c-exclusive driver line. Single T4 cells have elongated receptive fields that span multiple columns. When apparent motion consisting of ON illumination steps was focused on the center, both non-linear amplification of preferred direction sequences as well as non-linear suppression of null direction sequences could be observed, showing explicitly that both mechanisms underlie direction-selectivity on T4 dendrites. Critically, at the dorsal and ventral end of the receptive field only suppression and facilitation, respectively, occurred. This indicated a clear functional division among input elements at opposite ends of the dendritic tree.

In concerts, these studies argue in favor of a synthesis of Reichardt and Barlow-Levick models. The biological algorithm employs three visual inputs to implement both an excitatory and an inhibitory non-linearity. \citet{Haag:2016cq} put forward a simple extension of the standard correlation scheme in which the output of a multiplicative Reichardt sub-unit is divided by a third spatially displaced input, effectively creating a serial circuit consisting of two half-detectors. In this model, the central arm is a fast line and the flanking inputs contain appropriate delays. The model mimics T4 responses to apparent motion closely and produces plausible frequency tuning curves.

Importantly, it resolves a puzzling observation for T4 and T5 measurements that were part of our initial functional characterization \citep{Maisak:2013kk}. Sub-units of Reichardt detectors show little direction-selectivity by themselves; they generally respond vigorously to motion in both preferred and null direction. Only after subtraction do responses become cleanly sensitive to one or the other. Initial assumptions about the correspondence between model and circuit were that T4 and T5 are ON- and OFF-specific half-detectors as in the two-quadrant model by \citet{Eichner:2011ic}. However, both neurons exhibited remarkable selectivity for direction when stimulated with gratings or edges. That is, responses of T4 and T5 sub-types were sharply tuned to one direction in visual space and suffered from little off-target activation. The three-arm detector provides a consistent explanation for this property: if enhancement of preferred stimuli and suppression of non-preferred stimuli act in concert, crisp tuning follows. This architecture ensures high signal-to-noise ratio even at the local stage preceding subtractive and spatial integration. Given the data by \citet{Leong:2016hu} and subsequent work using telescopic stimulation (Jürgen Haag, personal communication), it is probable that motion extraction in T5 relies on a comparable dual non-linearity.

The elaboration of the correlation detector also contextualizes the input complexity that our work on T4 and T5 inputs has suggested \citep{Ammer:2015jo,Serbe:2016ew}. A motion detector of this type requires at least three arms, which reduces the number of seemingly extraneous input cells to just one. Moreover, given their relative spatial displacement, T4 inputs Mi1/Tm3, Mi4, and Mi9 now map neatly onto central and peripheral lines of this novel architecture. Taking this as well as temporal filter properties into account, \citet{Arenz:2017aa} could show through simulations that a detector in which Mi4 acts as a delayed facilitating input, Mi1 as a fast central facilitory input, and Mi9 as a delayed inhibitory input results in strong direction-selectivity as measured for orientations and grating frequencies. Finally, T4 inputs release a wide spectrum of neurotransmitters including glutamate, acetylcholine, and GABA \citep{Takemura:2017aa}. This offers many degrees of freedom for post-synaptic receptors to realize different non-linear interactions via synaptic machinery.

Due to a lack of dense reconstructions that trace processes of synaptic inputs, our understanding of how Tm1, Tm2, Tm4, and Tm9 synapses cluster on T5 dendrites is fundamentally limited \citep{Shinomiya:2014dx}. This adds difficulty to the mapping between cells and input lines of a three-arm detector. Quantitative work based on their filter properties finds that the best performing three-arm detectors have Tm2 as the fast central arm, Tm9 as the slow inhibitory arm, and either Tm1 or Tm4 as the excitatory arm \citep{Arenz:2017aa}. It is noteworthy that the OFF pathway appears to lack a second true low-pass filter next to Tm9. The resulting lack of kinetics differences poses a challenge when constructing two parallel correlation detectors. Curiously, all T5 inputs appear to be cholinergic. While it is possible that acetylcholine receptor diversity is sufficient to realize various non-linearities, further RNA profiling efforts may revise this picture in the future.

\subsubsection{Neural implementation}

% Three-arm detector (start with Fisher, go to Leong & Haag, mention neurotransmitters)
% Type of non-linear interaction
% -> Old Borst & Egelhaaf & Reichardt, Motion energy (squaring), Clark/Fitzg
% Biophysical implementation
% -> Koch, Torre, Gabbiani

% \subsection{Motion opponency}

% \subsection{Emergence of polarity-selectivity}
% % Dynamic vs step (stim depend)
% % Surprising completeness in lobula plate
% % Issues of DC
% % Yang
% % Relevant: messy lamina/medulla (new connectome...)

% \subsection{Sources of asymmetry}

% \subsection{Further elaborations}
% "Fourth line" -- modulatory...?
% Feedback
% Normalization -- mention Tm9?

% MAUSS STUFF!
% MAIMON STUFF!
% What about JoSe?
% % as well as even higher-resolving focused-ion beam scanning electron microscopy \citep[FIB-SEM;][]{Takemura:2015im}. The latter offers $\approx\SI{10}{\nano\meter}$ of isotropic voxel size.
% % Cite Takemura MOST FRESH
% \subsection{Multi-input detectors}
% \subsection{Implementing of the non-linearity}
% % % Cite Koch
% % % Mo & Koch, Torre & Poggio, Srinivasan (1976)
% \subsection{Source of asymmetries}
% \subsection{Emergence of polarity selectivity}
% % Yang, Clandinin

% \section{Comparative views}
% \subsection{Parallels in neural architecture}
% % Talk about cortex?
% \subsection{ON and OFF processing}
% \subsection{Asymmetries between ON and OFF}

% \section{Visual ecology of fruit flies}
% % What else can we compute from motion?

% \section{Behavior as a tool for sensory neuroscience}
% \subsection{Visuomotor transformation}
% % LPTCs as motor elements & state-dependency --> DISCUSSION
% STATE DEPENDEEENCYYYYY
% Mention Keller stuff!!!!

% \subsection{Limitations}
% \subsection{Advantages}
% Can use amplifying stuff to actually INCREASE blocks
% \subsection{Data mining}
% % Branson etc.
% vs interventionism as in the studies we did

% \section{Outlook}